% Description: Table Using CSV Files

% Note
% 1. In order to present a large amount of data in tables without writing such tables in Latex by hand, one can import
%    the data directly from .csv (comma-separated value) files.
% 2. LaTeX can generate tables from .csv files automatically.
% 3. To add new columns, simply duplicate the display column line and change the number and name.
% 4. Add new units using the \siunitx command and the ampersand seperator.
% 5. Have a .csv file seperated with comma as column seperator and newline as row seperator.
% 6. This will only work for tables smaller than one page.

% Note on CSV file
% 1. The CSV file must have comma as the column seperator.
% 2. The CSV file must have newline as the row seperator.
% 3. The column seperator and row seperator can be modified in the CSV files as well as in the code.
% 4. The numbers contained are actually longer than what shows up in the table because of rounding.

\documentclass{article}

\usepackage{booktabs}
\usepackage{siunitx}
\usepackage{pgfplotstable}                                              % Generates table from .csv

% Setup siunitx:
\sisetup{
  round-mode          = places,  % Rounds numbers
  round-precision     = 2,       % to 2 places
}

\begin{document}

\begin{table}[h!]
    \begin{center}
        \caption{Autogenerated table from .csv file.}
        \label{table1}
        \pgfplotstabletypeset[
            multicolumn names,                                          % allows to have multicolumn names
            col sep=comma,                                              % the seperator in our .csv file
             % The part that controls column names and formatting
            display columns/0/.style={
                column name=$Value 1$,                                  % name of first column
                column type={S},string type},                           % use siunitx for formatting
            display columns/1/.style={
                column name=$Value 2$,
                column type={S},string type},
            %%%%%%%%%%%%%%%%%%%%%%%%%%%%%%%%%%%%%%
            % Add more columns
            % display columns/2/.style={
		    % column name=$Value 3$,
		    % column type={S},string type},
            %%%%%%%%%%%%%%%%%%%%%%%%%%%%%%%%%%%%%%
            every head row/.style={
                before row={\toprule},                                  % have a rule at top
                after row={
                    \si{\ampere} & \si{\volt}\\                         % the units seperated by &
                    %%%%%%%%%%%%%%%%%%%%%%%%%%%%%%%%%%%%%%
                    % Add more columns - Add a new unit for the new column.
                    % \si{\ampere} & \si{\volt}& \si{\tesla}\\
                    %%%%%%%%%%%%%%%%%%%%%%%%%%%%%%%%%%%%%%
                    \midrule}                                           % rule under units
                },
                every last row/.style={after row=\bottomrule},          % rule at bottom
        ]{029-Tables_Data.csv}                                                 % filename/path to file
    \end{center}
\end{table}

\end{document}